% A LaTeX template for ARTICLE version of the MSc Thesis submissions to 
% Politecnico di Milano (PoliMi) - School of Industrial and Information Engineering
%
% S. Bonetti, A. Gruttadauria, G. Mescolini, A. Zingaro
% e-mail: template-tesi-ingind@polimi.it
%
% Last Revision: October 2021
%
% Copyright 2021 Politecnico di Milano, Italy. Inc. NC-BY

\documentclass[11pt,a4paper]{article} 

%------------------------------------------------------------------------------
%	REQUIRED PACKAGES AND  CONFIGURATIONS
%------------------------------------------------------------------------------
% PACKAGES FOR TITLES
\usepackage{titlesec}
\usepackage{color}

% PACKAGES FOR LANGUAGE AND FONT
\usepackage[utf8]{inputenc}
\usepackage[english]{babel}
\usepackage[T1]{fontenc} % Font encoding
\usepackage{FiraMono}
\usepackage{inconsolata}

% PACKAGES FOR IMAGES
\usepackage{graphicx}
\graphicspath{{Images/}}
\usepackage{eso-pic} % For the background picture on the title page
\usepackage{subfig} % Numbered and caption subfigures using \subfloat
\usepackage{caption} % Coloured captions
\usepackage{transparent}

% STANDARD MATH PACKAGES
\usepackage{amsmath}
\usepackage{amsthm}
\usepackage{bm}
\usepackage[overload]{empheq}  % For braced-style systems of equations

% PACKAGES FOR TABLES
\usepackage{tabularx}
\usepackage{longtable} % tables that can span several pages
\usepackage{colortbl}

% PACKAGES FOR ALGORITHMS (PSEUDO-CODE)
\usepackage{algorithm}
\usepackage{algorithmic}

% PACKAGES FOR REFERENCES & BIBLIOGRAPHY
\usepackage[colorlinks=true,linkcolor=black,anchorcolor=black,citecolor=black,filecolor=black,menucolor=black,runcolor=black,urlcolor=black]{hyperref} % Adds clickable links at references
\usepackage{cleveref}
\usepackage[square, numbers, sort&compress]{natbib} % Square brackets, citing references with numbers, citations sorted by appearance in the text and compressed
\bibliographystyle{plain} % You may use a different style adapted to your field

% PACKAGES FOR THE APPENDIX
\usepackage{appendix}

% PACKAGES FOR ITEMIZE & ENUMERATES 
\usepackage{enumitem}

% OTHER PACKAGES
\usepackage{amsthm,thmtools,xcolor} % Coloured "Theorem"
\usepackage{comment} % Comment part of code
\usepackage{fancyhdr} % Fancy headers and footers
\usepackage{lipsum} % Insert dummy text
\usepackage{tcolorbox} % Create coloured boxes (e.g. the one for the key-words)

%-------------------------------------------------------------------------
%	NEW COMMANDS DEFINED
%-------------------------------------------------------------------------
% EXAMPLES OF NEW COMMANDS -> here you see how to define new commands
\newcommand{\bea}{\begin{eqnarray}} % Shortcut for equation arrays
\newcommand{\eea}{\end{eqnarray}}
\newcommand{\e}[1]{\times 10^{#1}}  % Powers of 10 notation
\newcommand{\mathbbm}[1]{\text{\usefont{U}{bbm}{m}{n}#1}} % From mathbbm.sty
\newcommand{\pdev}[2]{\frac{\partial#1}{\partial#2}}
\newcommand{\m}[1]{{\fontfamily{fvm}\selectfont #1}}
\newcommand{\inc}[1]{{\fontfamily{zi4}\selectfont #1}}
% NB: you can also override some existing commands with the keyword \renewcommand

%----------------------------------------------------------------------------
%	ADD YOUR PACKAGES (be careful of package interaction)
%----------------------------------------------------------------------------


%----------------------------------------------------------------------------
%	ADD YOUR DEFINITIONS AND COMMANDS (be careful of existing commands)
%----------------------------------------------------------------------------


% Do not change Configuration_files/config.tex file unless you really know what you are doing. 
% This file ends the configuration procedures (e.g. customizing commands, definition of new commands)
\input{Configuration_files/config}

% Insert here the info that will be displayed into your Title page 
% -> title of your work
\renewcommand{\title}{Title}
% -> author name and surname
\renewcommand{\author}{Name Surname}
% -> MSc course
\newcommand{\course}{Xxxxxxxxxxxx Engineering - Ingegneria Xxxxxxxxxxxx}
% -> advisor name and surname
\newcommand{\advisor}{Prof. Name Surname}
% IF AND ONLY IF you need to modify the co-supervisors you also have to modify the file Configuration_files/title_page.tex (ONLY where it is marked)
\newcommand{\firstcoadvisor}{Name Surname} % insert if any otherwise comment
\newcommand{\secondcoadvisor}{Name Surname} % insert if any otherwise comment
% -> author ID
\newcommand{\ID}{Student ID}
% -> academic year
\newcommand{\YEAR}{20XX-20XX}
% -> abstract (only in English)
\renewcommand{\abstract}{Here goes the Abstract in English of your thesis (in article format)
followed by a list of keywords.
The Abstract is a concise summary of the content of the thesis (single page of text)
and a guide to the most important contributions included in your thesis.
The Abstract is the very last thing you write.
It should be a self-contained text and should be clear
to someone who hasn't (yet) read the whole manuscript.
The Abstract should contain the answers to the main research questions
that have been addressed in your thesis.
It needs to summarize the motivations and the adopted approach as well as
the findings of your work and their relevance and impact.
The Abstract is the part appearing in the record of your thesis inside POLITesi,
the Digital Archive of PhD and Master Theses (Laurea Magistrale) of Politecnico di Milano.
The Abstract will be followed by a list of four to six keywords.
Keywords are a tool to help indexers and search engines to find relevant documents.
To be relevant and effective, keywords must be chosen carefully.
They should represent the content of your work and be specific to your field or sub-field.
Keywords may be a single word or two to four words. }

% -> key-words (only in English)
\newcommand{\keywords}{here, the keywords, of your thesis}

%-------------------------------------------------------------------------
%	BEGIN OF YOUR DOCUMENT
%-------------------------------------------------------------------------
\begin{document}

%-----------------------------------------------------------------------------
% TITLE PAGE
%-----------------------------------------------------------------------------
% Do not change Configuration_files/TitlePage.tex (Modify it IF AND ONLY IF you need to add or delete the Co-advisors)
% This file creates the Title Page of the document
\input{Configuration_files/title_page}

\section{Introduzione}
\subsection{Specifiche}
Il componente riceve in ingresso una sequenza di \m{K} parole \m{W} il cui valore è compreso tra 0 e 255, dove il valore zero va interpretato come \textit{"valore non specificato"}. Le parole sono memorizzate a partire da un indirizzo \m{ADD} ogni due byte (e.g. \m{ADD, ADD+2... ADD+2*(K-1)}).
Il byte mancante (agli indirizzi \m{ADD+1, ADD+3..., ADD+2(K-1)+1}) dovrà essere completato con il valore di \textit{credibilità} \m{C} della relativa parola \m{W}: essa può assumere valori tra 31 e 0, è pari a 31 ogni volta che il valore \m{W} è \textit{valido} (diverso da zero) mentre viene decrementato rispetto al valore precedente nel caso in cui si incontri uno zero in \m{W}.\\
L'obiettivo del componente è processare la stringa e modificarla come segue:
\begin{itemize}
    \item[-] sostituire le parole con valore 0 con l'ultimo valore \textit{valido}, cioè l'ultimo valore della sequenza letto diverso da zero;
    \item[-] completare per ogni parola il byte mancante con il suo valore di \textit{credibilità} \m{C}.
\end{itemize} 

Un esempio di sequenza in ingresso è ad esempio (con valori in decimale):
\begin{table}[h]
    \centering
    \begin{tabular}{|>{\centering\arraybackslash}m{1cm}|>{\centering\arraybackslash}m{1cm}|>{\centering\arraybackslash}m{1cm}|>{\centering\arraybackslash}m{1cm}|>{\centering\arraybackslash}m{1cm}|>{\centering\arraybackslash}m{1cm}|>{\centering\arraybackslash}m{1cm}|>{\centering\arraybackslash}m{1cm}|>{\centering\arraybackslash}m{1cm}|>{\centering\arraybackslash}m{1cm}|}
        \hline
        \textbf{120} & 0 & \textbf{10} & 0 & \textbf{0} & 0 & \textbf{0} & 0 & \textbf{16} & 0 \\ \hline
    \end{tabular}
    \caption*{Stringa in ingresso con \m{K = 5} parole (in \textbf{grassetto})}
    
    \begin{tabular}{|>{\centering\arraybackslash}m{1cm}|>{\centering\arraybackslash}m{1cm}|>{\centering\arraybackslash}m{1cm}|>{\centering\arraybackslash}m{1cm}|>{\centering\arraybackslash}m{1cm}|>{\centering\arraybackslash}m{1cm}|>{\centering\arraybackslash}m{1cm}|>{\centering\arraybackslash}m{1cm}|>{\centering\arraybackslash}m{1cm}|>{\centering\arraybackslash}m{1cm}|}
        \hline
        \textbf{120} & \textit{31} & \textbf{10} & \textit{31} & \textbf{\textit{10}} & \textit{30} & \textbf{\textit{10}} & \textit{29} & \textbf{16} & \textit{31} \\ \hline
    \end{tabular}
    \caption*{Stringa processata (in \textit{corsivo} le modifiche)}
\end{table}

\textcolor{red}{Da completare con un esempio in grande}

\section{Architettura}
\subsection{Interfaccia e funzionamento}

Il modulo ha la seguente interfaccia:
\begin{itemize}
    \item tre segnali principali
    \begin{itemize}
        \item \m{i\_start}: 1 bit, segnale di inizio elaborazione;
        \item \m{i\_add}: 16 bit, indirizzo della memoria da cui inizia la porzione da processare;
        \item \m{i\_k}: 10 bit, numero di parole da elaborare;
    \end{itemize}
    \item due ingressi ausiliari
    \begin{itemize}
        \item \m{i\_clk}: 1 bit, segnale di clock unico per tutto il sistema;
        \item \m{i\_rst}: 1 bit, segnale di reset asincrono;
    \end{itemize}
    \item un'uscita principale
    \begin{itemize}
        \item \m{o\_done}: 1 bit, segnale di terminata elaborazione;
    \end{itemize}
    \item cinque connessioni con la memoria
    \begin{itemize}
        \item \m{i\_mem\_data}: 8 bit, contenuto in ingresso dalla memoria quando letta;
        \item \m{o\_mem\_en}: 1 bit, segnale di \textit{enable}, deve essere 1 per comunicare con la memoria;
        \item \m{o\_mem\_we}: 1 bit, segnale di \textit{write enable}, deve essere 1 per scrivere, 0 per leggere;
        \item \m{o\_mem\_addr}: 16 bit, indirizzo della memoria da cui leggere o in cui scrivere;
        \item \m{o\_mem\_data}: 8 bit, dato che verrà scritto in memoria;
    \end{itemize}
\end{itemize}
Tutti i segnali sono sincroni e devono essere interpretati sul fronte di salita del clock; fa eccezione il segnale di reset che è asincrono.
\begin{figure}[H]
    \centering
    \includegraphics[scale=.45]{export project.png}
    \caption*{Diagramma del componente}
\end{figure}

\end{document}